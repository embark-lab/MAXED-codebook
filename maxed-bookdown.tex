% Options for packages loaded elsewhere
\PassOptionsToPackage{unicode}{hyperref}
\PassOptionsToPackage{hyphens}{url}
%
\documentclass[
]{book}
\usepackage{amsmath,amssymb}
\usepackage{iftex}
\ifPDFTeX
  \usepackage[T1]{fontenc}
  \usepackage[utf8]{inputenc}
  \usepackage{textcomp} % provide euro and other symbols
\else % if luatex or xetex
  \usepackage{unicode-math} % this also loads fontspec
  \defaultfontfeatures{Scale=MatchLowercase}
  \defaultfontfeatures[\rmfamily]{Ligatures=TeX,Scale=1}
\fi
\usepackage{lmodern}
\ifPDFTeX\else
  % xetex/luatex font selection
\fi
% Use upquote if available, for straight quotes in verbatim environments
\IfFileExists{upquote.sty}{\usepackage{upquote}}{}
\IfFileExists{microtype.sty}{% use microtype if available
  \usepackage[]{microtype}
  \UseMicrotypeSet[protrusion]{basicmath} % disable protrusion for tt fonts
}{}
\makeatletter
\@ifundefined{KOMAClassName}{% if non-KOMA class
  \IfFileExists{parskip.sty}{%
    \usepackage{parskip}
  }{% else
    \setlength{\parindent}{0pt}
    \setlength{\parskip}{6pt plus 2pt minus 1pt}}
}{% if KOMA class
  \KOMAoptions{parskip=half}}
\makeatother
\usepackage{xcolor}
\usepackage{longtable,booktabs,array}
\usepackage{calc} % for calculating minipage widths
% Correct order of tables after \paragraph or \subparagraph
\usepackage{etoolbox}
\makeatletter
\patchcmd\longtable{\par}{\if@noskipsec\mbox{}\fi\par}{}{}
\makeatother
% Allow footnotes in longtable head/foot
\IfFileExists{footnotehyper.sty}{\usepackage{footnotehyper}}{\usepackage{footnote}}
\makesavenoteenv{longtable}
\usepackage{graphicx}
\makeatletter
\def\maxwidth{\ifdim\Gin@nat@width>\linewidth\linewidth\else\Gin@nat@width\fi}
\def\maxheight{\ifdim\Gin@nat@height>\textheight\textheight\else\Gin@nat@height\fi}
\makeatother
% Scale images if necessary, so that they will not overflow the page
% margins by default, and it is still possible to overwrite the defaults
% using explicit options in \includegraphics[width, height, ...]{}
\setkeys{Gin}{width=\maxwidth,height=\maxheight,keepaspectratio}
% Set default figure placement to htbp
\makeatletter
\def\fps@figure{htbp}
\makeatother
\setlength{\emergencystretch}{3em} % prevent overfull lines
\providecommand{\tightlist}{%
  \setlength{\itemsep}{0pt}\setlength{\parskip}{0pt}}
\setcounter{secnumdepth}{5}
\usepackage{booktabs}
\usepackage{amsthm}
\makeatletter
\def\thm@space@setup{%
  \thm@preskip=8pt plus 2pt minus 4pt
  \thm@postskip=\thm@preskip
}
\makeatother
\ifLuaTeX
  \usepackage{selnolig}  % disable illegal ligatures
\fi
\usepackage[]{natbib}
\bibliographystyle{apalike}
\usepackage{bookmark}
\IfFileExists{xurl.sty}{\usepackage{xurl}}{} % add URL line breaks if available
\urlstyle{same}
\hypersetup{
  pdftitle={Multimodal Assessment of Exercise in Eating Disorders (MAXED) - Study Procedures and Measures},
  pdfauthor={Katherine Schaumberg, Sabrina Liu},
  hidelinks,
  pdfcreator={LaTeX via pandoc}}

\title{Multimodal Assessment of Exercise in Eating Disorders (MAXED) - Study Procedures and Measures}
\author{Katherine Schaumberg, Sabrina Liu}
\date{2025-03-17}

\begin{document}
\maketitle

{
\setcounter{tocdepth}{1}
\tableofcontents
}
\chapter{Introduction}\label{introduction}

This codebook contains study information, measures, and complete data dictionary for the Multimodal Assesment of Exercise in Eating Disorders (MAXED) Study

\chapter{Study Premise}\label{study-premise}

Project MAXED (Phase 1: 2020-2024) is a study is conducting and investigation of exercise response among girls and young women with eating disorders developed by the EMBARK lab in 2020.

MAXED R21MH131787 (Phase 2: 2024- ) is being conducted at the University of California San Francisco (UCSF) and the University of Wisconsin-Madison and is a continuation of the MAXED study with a few protocol changes.

1.1 Specific Aims (MAXED):

Driven exercise is a serious and common feature of eating disorders, but current understanding of factors that give rise to and maintain driven exercise is limited.
Project MAXED aims to evaluate acute psychobiological response to a bout of moderate-intensity exercise among young females (age 14-22 years) with and without mild-to-moderate eating disorder symptoms.
Overall, this study will contribute to the conceptualization of driven exercise and how individuals' acute biological and affective responses to exercise contribute to risk for and maintenance of driven exercise.
This pilot study has two primary aims:

Aim 1: Confirm feasibility of paradigms evaluating acute response to exercise among outpatient individuals with mild-to-moderate eating disorder symptoms.
We will confirm feasibility of our exercise-based tasks via a) study dropout at all timepoints, b) adverse events, c) completion rates of study tasks.
Over the course of the study, we expect both eating disorder and healthy control groups to meet thresholds of \textless{} 20\% dropout, zero adverse events, and \textgreater{} 80\% task completion.

Aim 2: Characterize variability in biobehavioral response to in-lab exercise among individuals with mild-to-moderate eating disorders.
We will characterize changes during exercise in state body image, mood, and biological markers in both eating disorder and healthy control groups; we will specifically characterize mean levels of, and variability in, biobehavioral response to exercise across the groups.

1.2 Specific Aims (MAXED R21MH131787):

This pilot R21 project will: 1) characterize variability in psychobiological response to in-lab moderate-intensity exercise among those with and without EDs; 2) create and refine exercise tasks that capture in-vivo exercise response; and 3) test the hypothesis that acute exercise response relates to DEx in those with EDs.
Our sample will include 67 female subjects (aged 16-22y) with a restrictive-spectrum ED diagnosis (AN; atypical AN; AN-spectrum OSFED) who are medically cleared for moderate-intensity exercise, along with 33 age-matched healthy controls (HC).
We will conduct a preliminary examination of multi-modal response (self-report, task-based learning, neurotransmitter shifts) to acute exercise using two novel tasks where subjects will: (Task 1) drink a high-calorie beverage prior to a self-paced bout of exercise to trigger threat and capture threat-reduction mechanisms of exercise, and (Task 2) engage in a prescribed, controlled bout of exercise to assess psychobiological exercise response.

Aim 1: Internal validation of in-lab exercise tasks: characterize variability in psychobiological response to in-lab exercise across ED and HC.
H1a: We expect improvements in both affect and body image during exercise, significant increases in cortisol, eCBs, BDNF, and significant decreases in leptin following moderate-intensity aerobic exercise, and no change in circulating biomarkers following the rest condition among those with EDs.
H1b: Individuals with EDs will exhibit stronger biomarker, body image (+), and affective (+) response to exercise as compared to HCs.
Exploratory: whether exercise impacts on learning differ across HC and ED.
H1c: Acute exercise response will evidence greater variability among those with EDs compared to HC.

Aim 2: External validation of in-lab acute exercise tasks: examine associations between acute exercise response and ED severity, self-reported DEx, and accelerometer-measured moderate-to-vigorous physical activity (MVPA) across ED and HCs.
H2: Findings will support concurrent validity of tasks.
Acute exercise response will positively associate with ED severity and DEx among those with EDs, and positively associate with free-living MVPA across ED and HCs.
Detailed explication of acute exercise effects among individuals with EDs in a controlled setting will (i) improve assessment of DEx risk and function among those with EDs, (ii) elucidate a testable model of DEx, and (iii) suggest targets for mechanistically informed DEx intervention

\chapter{Procedures}\label{procedures}

The first wave of this study (MAXED) will include recruitment of 40 (20 control and 20 participants with ED symptoms) female adolescents and young adults aged 14-22.

The second wave of this study (MR21) will include the recruitment of 100 (50 control and 50 participants with ED symptoms) female adolescents and young adults aged 16-22.

\section{Recruitment}\label{recruitment}

Recruitment sources include: (1) University Health Services at UW-Madison and other area colleges and universities (2) flyers at area businesses (e.g.~fitness centers, coffee shops) (3) local mental health providers, including providers within Wisconsin Psychiatric Institute and Clinic (WisPIC), as well as outside of the UW settings (e.g., at Community Mental Health Centers, private practices, etc.) (4) online advertisement on social media platforms (5) Personal recruitment by study staff, (6) campus e-mail recruitment.

\chapter{Interventions}\label{interventions}

There are 2 stages to this protocol: 1) Screening Day with screening assessments; and 2) Day A, DayB, and Day C that consist of computational and exercise tasks, questionnaires, and blood draw. Participants will complete these two stages across 4 separate days. Day A, Day B, and Day C will occur randomly, spaced 6-8 days apart.

\section{Screening Day stage}\label{screening-day-stage}

During the Screening Day stage, participants will undergo a structured clinical interview to assess current eating disorder symptoms, psychological symptoms,cognitive function assessment, physical status, and demographics. If participants are under 18 years old, parents will also be asked to complete a demographics form. Participants will also complete the Approach Avoidance Conflict task. The assessment stage lasts 2-3 hours.

\section{Day A, Day B, and Day C Stage}\label{day-a-day-b-and-day-c-stage}

During Day A, participants will complete two different tasks--Task 1: ``Reward task'' (rest condition) will include self-report measures, two blood draws immediately before and after completing 30 minutes of rest to examine changes in key neurotransmitters over time, and a post-rest behavioral task; and Task 2: ``Threat task''(active condition) will include self-report of state emotional, body image, behavioral urges and cravings,a food challenge task of drinking a milkshake, and 30 minutes of exercise.

During Day B, participants will complete Task 1: Reward (active condition), that includes self-report questionnaires, a behavioral task in which participants will work to earn ``points'' for exercise, and two blood draws, including one immediately before and one immediately after completing the 30 minutes of exercise to examine changes in key neurotransmitters.

On Day C, participants will complete Task 2: Threat (rest condition),which will include completion of self-report questionnaires, a food challenge task, and 30 minutes of rest. Participants will provide four blood samples total during this study. In order to ensure that blood monoamine and hormone levels are controlled, participants will be asked to refrain from eating after1:00pm, only eating a provided nutritional supplement (nutrition bar) as a snack before attending Day A and Day B study visits at 4pm.

Additionally, Task Day B will be scheduled during the follicular phase to control for possible menstrual phase effects on the biological response to exercise. Menstrual phase will be further clarified and controlled by the collection of blood samples on Task Day B to quantify estradiol levels. Participants will also complete a reinforcement learning task following Day A and Day B to examine the acute impact of exercise on reinforcement learning mechanisms. Day A, Day B, and Day C will each last about 2 hours. Lastly, participants will be provided with a fitness tracker to wear for 7 days between the screening visit and their second visit to objectively determine average levels of physical activity.

\chapter{Self-Report Measures}\label{self-report-measures}

\section{Compulsive Exercise Test (CET)}\label{compulsive-exercise-test-cet}

The Compulsive Exercise Test measures cognitive, affective, and
behavioral components of excessive exercise in individuals (Taranis et
al., 2011). There are 24 items that contain 5 subscales which are
avoidance and rule-driven behavior, weight control exercise, mood
improvement, lack of exercise enjoyment, and exercise rigidity. Each
statement is rated on a Likert scale of 1 (never true) to 5 (always
true), and the score of each subscale is determined by averaging the
items for that factor. Higher scores indicate greater pathology of
excessive exercise.

\textbf{Scoring} Scoring for the CET includes: 1. selects only variables that
are relevant for the current measure

\begin{enumerate}
\def\labelenumi{\arabic{enumi}.}
\setcounter{enumi}{1}
\item
  Recode all variables (e.g.~changing ``never true = 1'' to ``never true
  = 0''), renamed necessary variables (e.g.~cet\_week\_repeat to
  cet\_week)
\item
  Creates a symptom sum score, which gives a count of the number of
  compulsive exercise symptoms (0-5) that are present for each
  individual
\end{enumerate}

\textbf{Key Variables} \texttt{cet\_sum} (defines the severity of compulsive exercise
based on number of symptoms)

\section{Childhood Trauma Questionnaire (CTQ)}\label{childhood-trauma-questionnaire-ctq}

The Childhood Trauma Questionnaire (CTQ; Bernstein et al., 1994) is a
70-item questionnaire designed to measure multiple aspects of trauma in
childhood. However, we have adapted this questionnaire to only include
28 items. Each item asks the individual to respond to a specific
question following the prompt, ``When I was growing up\ldots{}'', and allows
individuals to respond on a 5-point scale ranging from 1 (``Never True'')
to 5 (``Very Often True''). There are five sub-scales in this
questionnaire: emotional abuse, physical abuse, sexual abuse, emotional
neglect, and physical neglect. There were originally four, physical and
emotional abuse were combined, but we chose to split them up for better
accuracy. Each sub-scale is summed individually. High sub-scale scores
indicate more childhood trauma in that sub-scale, and low scores
indicate low childhood trauma in that sub-scale.

\textbf{Scoring} 1. Selects only variables that are relevant for the current
measure

\begin{enumerate}
\def\labelenumi{\arabic{enumi}.}
\setcounter{enumi}{1}
\item
  Recode all necessary variables (e.g.~`Never true =1' to `never true
  = 0')
\item
  Creates a symptom sum score, which gives a sum count of the number
  of symptoms (0-5) that are present for each individual
\end{enumerate}

\textbf{Key Variables} \texttt{ctq\_sum} (defines sum score of symptoms)

\section{Drive for Muscularity (DFM)}\label{drive-for-muscularity-dfm}

The Drive for Muscularity Scale is a 15 item questionnaire that assesses
the perceptions and behaviors surrounding the desire to gain muscles
(McCreary \& Sasse, 2000). Questions are rated on a Likert scale from 1
(Always) to 6 (Never). The items are then reversed scored and higher sum
scores indicating higher drive for muscularity.

\textbf{Scoring} 1. Selects only variables that are relevant for the current
measure

\begin{enumerate}
\def\labelenumi{\arabic{enumi}.}
\setcounter{enumi}{1}
\item
  Items are re-coded to be reversed coded (``1=5, 2=4, 3=3, 4=2, 5=1,
  6=0'')
\item
  Creates a symptom average score, which gives an average count of the
  number of symptoms (0-5) that are present for each individual
\end{enumerate}

\textbf{Key Variables} \texttt{muscularity\_average} (defines sum score of symptoms)

\section{Intolerance of Uncertainty (IUS)}\label{intolerance-of-uncertainty-ius}

The IUS-S is a 12 item scale that Carleton et al.~(2007) adapted from
the original 27-item Intolerance of Uncertainty Scale from Freeston et
al.~(1994). It assesses the worry that an individual has about the
possibility of negative events or outcomes, and there are two subscales
that address anxious and avoidant aspects of the intolerance. Each item
is scored using a 5-point Likert scale, and the scores are summed, with
higher scores signifying a greater intolerance of uncertainty.

\textbf{Scoring} 1. Selects only variables that are relevant for the current
measure

\begin{enumerate}
\def\labelenumi{\arabic{enumi}.}
\setcounter{enumi}{1}
\item
  Recoded all necessary variables (e.g.~changing ``not at all
  characteristic of me = 1'' to ``not characteristic of me = 0'')
\item
  Creates a symptom average score, which gives an average count of the
  number of symptoms (0-5) that are present for each individual
\end{enumerate}

\textbf{Key Variables} \texttt{iuss\_sum} (defines average score of symptoms)

\section{The Mini Mental State Examination (MMSE)}\label{the-mini-mental-state-examination-mmse}

The Mini Mental State Examination (MMSE) assesses cognitive ability and
is frequently used to screen for dementia and examine the severity and
progression of cognitive impairment (Kurlowicz \& Wallace, 1999). The
five categories measured are orientation, registration, attention and
calculation, recall, and language. There are 11 items on the
examination, and the scores range from 0-30, with a score 23 or below
signaling cognitive impairment.

\textbf{Scoring} 1. selects only the variables that are relevant for the
current measure

\begin{enumerate}
\def\labelenumi{\arabic{enumi}.}
\setcounter{enumi}{1}
\tightlist
\item
  Renames raw variables
\end{enumerate}

\textbf{Key Variables} \texttt{mmse\_total} (sum of a participant's score)

\section{Brief Fear of Negative Evaluation Scale (BFNE)}\label{brief-fear-of-negative-evaluation-scale-bfne}

Leary (1983) shortened the Fear of Negative Evaluation Scale (Watson \&
Friend, 1969) to create the BFNE. It measures the tolerance to the
possibility of judgment from others. There are 12 items that use a
Likert scale (1 `Not at all' to 5 `Extremely') to rate how
characteristic each statement is of the respondent. The items are summed
to create a total score where higher scores indicate greater fear of
negative evaluation.

\textbf{Scoring} 1. Selects raw variables being used for the current measure

\begin{enumerate}
\def\labelenumi{\arabic{enumi}.}
\setcounter{enumi}{1}
\item
  Re-code variables to new variable names and values (e.g.~`Not at all
  characteristic of me =1' to `Not at all characteristic of me =0')
\item
  Sum the total scores
\end{enumerate}

\textbf{Key Variables} \texttt{bfnes\_sum}

\section{BIS/BAS Scale}\label{bisbas-scale}

The Behavioral Activation System (BAS) and Behavioral Inhibition System
(BIS) scales were developed by Carver and White (1994) to assess how
individuals respond to situations. Each statement is rated on a 4-point
scale of how strongly one thinks that the statement applies to
themselves. The BAS, which is the extent to which someone acts to gain
rewards or positive outcomes, has three sub-scales: reward
responsiveness, drive and fun seeking. There is one sub-scale for BIS,
which is the extent to which someone acts to avoid negative outcomes.

\textbf{Scoring} 1. Selects all relevant variables

\begin{enumerate}
\def\labelenumi{\arabic{enumi}.}
\setcounter{enumi}{1}
\tightlist
\item
  Adds variables \texttt{bis\_ambitious} \texttt{bis\_all\_out} \texttt{bis\_act\_now}
  \texttt{bis\_no\_hold} get \texttt{bas\_drive}.
\item
  Adds variables \texttt{bis\_explore} \texttt{bis\_fun} \texttt{bis\_spontaneous}
  \texttt{bis\_crave\_excite} to get \texttt{bas\_fun\_seeking}
\item
  Adds variables \texttt{bis\_love\_doing} \texttt{bis\_excitement}
  \texttt{bis\_opportunity\_excite} \texttt{bis\_positive\_effect} \texttt{bis\_excite\_win} to
  get \texttt{bas\_reward\_responsiveness}
\item
  Adds variables \texttt{bis\_negative\_event\_fear} \texttt{bis\_criticism}
  \texttt{bis\_angry\ bis\_worked\_up} \texttt{bis\_worry\_poor\_perform} \texttt{bis\_no\_fear}
  \texttt{bis\_worry\_for\_mistake} to get \texttt{bis\_sum}
\end{enumerate}

\textbf{Key Variables} \texttt{bis\_sum} \texttt{bas\_drive} \texttt{bas\_fun\_seeking}
\texttt{bas\_reward\_responsiveness}

\section{The Body Image States Scale (BISS)}\label{the-body-image-states-scale-biss}

The Body Image States Scale (BISS) is a six-item measure of individuals'
evaluation and affect about their physical appearance at a particular
moment in time. They score from 0 (least impaired) to 8 (most impaired).

\textbf{Scoring} 1. Selects raw variables being used for the current measure

\begin{enumerate}
\def\labelenumi{\arabic{enumi}.}
\setcounter{enumi}{1}
\item
  Renames variables to be easily identified
\item
  Recode variables so that the least impaired = 0 and the most
  impaired = 8
\item
  Sum the total scores and rename this summary as biss\_sum
\end{enumerate}

\textbf{Key Variables} \texttt{biss\_appearance\_pre} \texttt{biss\_body\_size\_pre}
\texttt{biss\_weight\_pre} \texttt{biss\_attractive\_pre} \texttt{biss\_looks\_pre}
\texttt{biss\_average\_looks\_pre}

\section{The Difficulties in Emotion Regulation Scale (DERS)}\label{the-difficulties-in-emotion-regulation-scale-ders}

The Difficulties in Emotion Regulation Scale (DERS) is an instrument
measuring emotion regulation problems developed by K.L. Gratz and L.
Roemer.The self-report scale is comprised of 36 items asking respondents
how they relate to their emotions in order to produce scores on 6
different subscales.This tool can be especially useful in helping
patients identify areas for growth in how they respond to their
emotions, especially those with Borderline Personality Disorder,
Generalised Anxiety Disorder or Substance Use Disorder. The DERS scale
has been shown to have high internal consistency, good test--retest
reliability, and adequate construct and predictive validity (Gratz \&
Roemer, 2003).

\textbf{Scoring}

\textbf{Key Variables}

\section{Eating Disorder Diagnostic scale (ED History)}\label{eating-disorder-diagnostic-scale-ed-history}

Eating Disorder Diagnostic scale, which is a 22-item self-report
questionnaire designed to measure Anorexia nervosa, Bulimia nervosa, and
Binge-eating disorder symptomatology aligned with the DSM-IV diagnostic
criteria. The scale is comprised of a combination of Likert ratings,
dichotomous scores, behavioural frequency scores, and open-ended
questions asking for weight and height.

\textbf{Scoring} 1. Selects raw variables being used for the current measure

\begin{enumerate}
\def\labelenumi{\arabic{enumi}.}
\setcounter{enumi}{1}
\item
  Renames variables to be easily identified
\item
  Sum the total scores and rename this summary as edhistory\_sum
\end{enumerate}

\textbf{Key Variables} \texttt{edhistory\_weightloss} \texttt{edhistory\_fear\_fat}
\texttt{edhistory\_feel\_fat} \texttt{edhistory\_thin} \texttt{edhistory\_danger}
\texttt{edhistory\_limit\_food} \texttt{edhistory\_concentrate} \texttt{edhistory\_binge}
\texttt{edhistory\_not\_hungry} \texttt{edhistory\_alone} \texttt{edhistory\_guilt}
\texttt{edhistory\_upset} \texttt{edhistory\_self\_vomit} \texttt{edhistory\_laxatives}
\texttt{edhistory\_diuretics} \texttt{edhistory\_fast}

\section{Food Cravings Questionnaire(FCQ)}\label{food-cravings-questionnairefcq}

Food Cravings Questionnaire which is used instrument to assess food
cravings as a multidimensional construct. Its 39 items have an
underlying nine-factor structure to demonstrate food cravings as well as
restrictions.

\textbf{Scoring} 1. Selects raw variables being used for the current measure

\begin{enumerate}
\def\labelenumi{\arabic{enumi}.}
\setcounter{enumi}{1}
\item
  Renames variables to be easily identified
\item
  Recode variables so that ``strongly disagree'' = 0 and ``strongly
  agree'' = 4
\item
  Sum the total scores and rename this summary as fcq\_sum
\end{enumerate}

\textbf{Key Variables} \texttt{fcq\_desire\_restrict\_pre} \texttt{fcq\_desire\_fast\_pre}
\texttt{fcq\_desire\_vomit\_pre} \texttt{fcq\_desire\_laxatives\_pre}
\texttt{fcq\_desire\_exercise\_pre} \texttt{fcq\_desire\_binge\_pre}

\section{Frost Multidimensional Perfectionism Scale (FMPS)}\label{frost-multidimensional-perfectionism-scale-fmps}

The Frost Multidimensional Perfectionism Scale (FMPS) is a 35 question
self-report measure with four sub-scales of perfectionism. It contains a
total of 35 items. These are subsumed to the following, originally six,
now four subscales: Concern over mistakes and doubts about actions,
Excessive concern with parents' expectations and evaluation, Excessively
high personal standards, Concern with precision, order and organisation.
Each item is scored on a 5-point Likert-style scale (1 = strongly
disagree; 5= strongly agree) to describe how well each item describes
the participant experiences. Scores are derived by summing responses
across the questions included in each subscale. High scores on the
Organization subscale do not contribute to Total Perfectionism and are
not intrinsically problematic, but combined with high scores on the
other factors may exacerbate dysfunction.

\textbf{Scoring} 1. selects only the variables that are relevant for the
current measure

\begin{enumerate}
\def\labelenumi{\arabic{enumi}.}
\setcounter{enumi}{1}
\item
  creates six additional variables based on sum scores reflecting six
  subscales of the questionnaire: m. It contains a total of 35 items.
  These are subsumed to the following, originally six, now four
  subscales: Concern over mistakes and doubts about actions, Excessive
  concern with parents' expectations and evaluation, Excessively high
  personal standards, Concern with precision, order and organisation
\item
  select only a few columns that will go into the final dataset
\end{enumerate}

\textbf{Key Variables} \texttt{fmps\_concerns\_mistakes} (reflects participant's
concern over mistakes and doubts about actions)

\texttt{fmps\_concerns\_parents\_expectations} (reflects participant's excessive
concern with parents' expectations and evaluation)

\texttt{fmps\_high\_personal\_standards} (reflects participant's excessively high
personal standards)

\texttt{fmps\_concerns\_precision\_order} (reflects participant's Concern with
precision, order and organisation )

\texttt{fmps\_total\_perfectionism\_score} (reflects participant's total
perfectionism scores)

\section{Functions of Exercise Scale (FOE)}\label{functions-of-exercise-scale-foe}

The Functions of Exercise Scale was developed by Patricia Marten
DiBartolo, Linda Lin, Simone Montoya, Heather Neal, and Carey Shaffer.
The scale includes two subscales: Weight and Appearance (WA), and Health
and Enjoyment (HE). The FES is a 16-item, self-report questionnaire that
assesses motivation to exercise. Individuals provide ratings using a
7-item scale from ``1 = do not agree'' to ``7 = strongly agree''. FES-HE
scores are positively correlated with psychological well-being and
physical health. Conversely, FES-WA scores are negatively correlated
with depressive and eating disorder symptoms, self-esteem, and physical
well-being.

\textbf{Scoring}

\textbf{Key Variables}

\section{Menstrual Cycle Information (MCI)}\label{menstrual-cycle-information-mci}

The Menstrual Cycle Information is a form of retrospective
questionnaires (rating severity of symptoms from memory) that examines
the participant's menstrual information and secondary sexual
characteristics. It consists 22 questions, including open-ended, yes-no,
and

\textbf{Scoring} 1. selects only the variables that are relevant for the
current measure

\begin{enumerate}
\def\labelenumi{\arabic{enumi}.}
\setcounter{enumi}{1}
\item
  rename raw variables to appropraite names that are easy to
  understand
\item
  recode old variables to make it consistent that no equals to zero in
  the scoresheet
\item
  select only a few columns that will go into the final dataset
\end{enumerate}

\textbf{Key Variables} \texttt{mci\_estimate} (assess whether participant can
reliably estimate the stages of her cycle)

\texttt{mci\_public\_hair} (reflects participant's public hair development)

\texttt{mci\_hysterectomy} (assess whether participant has had a hysterectomy)

\section{NVS Self-report}\label{nvs-self-report}

The NVS Self-report states questionnaire consists three different parts:
the first four questions measuring mental efforts, then six questions
assessing body image states, and the last eighteen questions examining
food craving intentions.

\textbf{Scoring} 1. selects only the variables that are relevant for the
current measure

\begin{enumerate}
\def\labelenumi{\arabic{enumi}.}
\setcounter{enumi}{1}
\item
  rename raw variables to appropraite names that are easy to
  understand
\item
  creates three additional variables based on sum scores reflecting
  three components of the questionnaire: mental efforts, body image
  states, and food craving. Meanwhile, recode old variables to make it
  consistent that ``strongly disagree'' and ``extremely dissatisfied''
  equal to zero
\item
  select only a few columns that will go into the final dataset
\end{enumerate}

\textbf{Key Variables} \texttt{nvs\_mental\_effort} (reflects participant's mental
effort scores)

\texttt{nvs\_body\_image} (reflects participant's body image satisfication)

\texttt{nvs\_food\_craving} (reflects participant's food craving intents)

\section{Physical activity affect scale (PAAS)}\label{physical-activity-affect-scale-paas}

\textbf{Scoring} 1. Selects raw variables being used for the current measure

\begin{enumerate}
\def\labelenumi{\arabic{enumi}.}
\setcounter{enumi}{1}
\item
  Renames variables to be easily identified
\item
  Sum the total scores and rename this summary as biss\_sum
\end{enumerate}

\textbf{Key Variables} \texttt{paas\_enthusiastic\_pre} \texttt{paas\_crummy\_pre}
\texttt{paas\_fatigued\_pre} \texttt{paas\_calm\_pre}

\section{State Trait Anxiety Inventory is a self-evaluation (STAI)}\label{state-trait-anxiety-inventory-is-a-self-evaluation-stai}

The State Trait Anxiety Inventory is a self-evaluation questionnaire
developed by Charles D. Spielberger. It can be used in clinical settings
to diagnose anxiety and to distinguish it from depressive syndromes.
Form Y, its most popular version, has 20 items for assessing trait
anxiety and 20 for state anxiety. All items are rated on a 4-point
scale, and higher scores indicate greater anxiety.

\textbf{Scoring}

\textbf{Key Variables}

\section{Yale-Brown Obsessive Compulsive Scale(YBOCS)}\label{yale-brown-obsessive-compulsive-scaleybocs}

The Yale-Brown Obsessive Compulsive Scale was developed by Wayne Goodman
Dennis Charney, and is designed to rate the types of symptoms in
patients with Obsessive Compulsive Disorder and their severity. This
rating scale is intended for use as a semi-structured interview. The
interview should assess the items in the listed order and use the
questions provided. The total score is usually computed from the
subscales for obsessions (items 1-5) and compulsions (items 6-10).

\textbf{Scoring}

\textbf{Key Variables}

\chapter{Task Measures}\label{task-measures}

\chapter{Sample Description}\label{sample-description}

Recruitment sources include: (1) University Health Services at UW-Madison and other area colleges and universities (2) flyers at area businesses (e.g.~fitness centers, coffee shops) (3) local mental health providers, including providers within Wisconsin Psychiatric Institute and Clinic (WisPIC), as well as outside of the UW settings (e.g., at Community Mental Health Centers, private practices, etc.) (4) online advertisement on social media platforms (5) direct, peer-to-peer recruitment by peer facilitators, (6) campus e-mail recruitment.

\chapter{Data Requests and Terms of Use}\label{data-requests-and-terms-of-use}

We aim to share data from the MAXED project will all interested parties.

Depending on data necessary for analysis, this may include providing collaborative access to the data via external log-in to our \href{https://kb.wisc.edu/researchdata/page.php?id=96642}{research drive} and completing a \href{https://rsp.wisc.edu/contracts/DTUA-De-IdentifiedData.pdf}{data transfer and use agreement}. We expect most data requests will take 2-4 weeks for administrative processing.

In order to request data use, please complete the following steps:

\textbf{1.} Send an email with the title `MAXED data use' to \href{mailto:embarklab@psychiatry.wisc.edu}{\nolinkurl{embarklab@psychiatry.wisc.edu}} with the following information:

\textbf{a.} Name, affiliation, training status and year (e.g.~second-year graduate student; post-baccalaureate research coordinator, first-year postdoc, assistant professor)

\textbf{b.} Tentative title for proposed analysis and 1-paragraph description

\textbf{c.} Two identified EMBARK Lab collaborators you would be interested in working with (see \href{https://embark.psychiatry.wisc.edu/index.php/people/}{lab personnel} for information about lab members and their interests)

\textbf{2.} \href{https://calendly.com/katherine-schaumberg/15min}{Book a 15-minute meeting} with Dr.~Schaumberg to discuss your idea.

For secondary analysis projects that \emph{exclusively} use MAXED data, Dr.~Schaumberg will serve as senior (last) author, and at least one additional EMBARK Lab collaborator will join as an author on the project. Alternative arrangements can be discussed if mAXED is one of multiple data sources for a project.

  \bibliography{MAXED\_bookdown.bib,packages.bib}

\end{document}
